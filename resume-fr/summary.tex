%-------------------------------------------------------------------------------
%	SECTION TITLE
%-------------------------------------------------------------------------------
\cvsection{Présentation}


%-------------------------------------------------------------------------------
%	CONTENT
%-------------------------------------------------------------------------------
\begin{cvparagraph}

%---------------------------------------------------------
Développeur Web depuis l'arrivée d'Internet en Belgique (il y a 20 ans, quand Microsoft FrontPage et Adobe Dreamweaver étaient les seuls outils disponibles). J'ai vite étendu mes connaissances vers des langages dynamiques tels que le PHP avec lequel j'ai développé des sites dynamiques communément appelés de nos jours CMS. À cette époque, WordPress n'existait pas encore.

Je dispose d'un baccalauréat en informatique de gestion. J'aime constamment améliorer mes connaissances et mon expérience. Apprendre de nouvelles choses n'est pas un souci pour moi.

Durant ma carrière j'ai eu l'occasion  d'analyser, architecturer, développer, mettre en production, maintenir et apporter du support utilisateur pour de multiples projets. Ces derniers étant basés sur de nombreuses technologies telles que Java/Python/HTML/CSS/JavaScript/SQL/InterSystems Caché et le standard HL7 pour les données médicales. Des exemples plus détaillés se trouvent dans la section \emph{Projets}.


Travailler dans le milieu médical pendant 13 ans m'a permis à plusieurs reprises de travailler sur des interfaces d'intégration HL7. Au cours des projets, j'ai dû analyser, transformer, intégrer et échanger des données de santé basées sur le standard HL7 v2.x, aussi bien à l'aide de logiciels et scripts développés en interne que par la manipulation de la plateforme InterSystems HealthShare.

Je suis habitué à travailler sur plusieurs projets simultanément tout en m'adaptant à leurs fréquentes évolutions et changements de priorité.


J'aime développer et mettre en place des solutions de grande qualité pour les utilisateurs.
Pour moi, le but de l'informatique est de rendre la vie des gens plus simple.
\end{cvparagraph}
