%-------------------------------------------------------------------------------
%	SECTION TITLE
%-------------------------------------------------------------------------------
\cvsection{Projets}


%-------------------------------------------------------------------------------
%	CONTENT
%-------------------------------------------------------------------------------
\begin{cventries}


    
%---------------------------------------------------------
  \cventry
    {2019 - Présent} % Degree
    {Équipe de développement de modules Xperthis Care} % Institution
    {} % Location
    {} % Date(s)
    {
    Xperthis Care est un magnifique logiciel médical. Cependant certaines fonctionnalités utiles à l'équipe médicale sont manquantes.
    Nous avons dû développer des modules complémentaires dans le langage de programmation InterSystems Caché et Angular afin de répondre aux besoins et attentes des utilisateurs.

    L'un des modules concerne l'intégration de l'agenda du médecin dans le logiciel Xperthis Care pour que le praticien ne doive pas permuter entre plusieurs logiciels. J'ai eu l'occasion d'analyser, architecturer et développer ce projet.

    Technologies utilisées : InterSystems Caché / HTML / CSS / Angular / TypeScript / GitLab CI.

    }
    
    
%---------------------------------------------------------
  \cventry
    {2019 - Présent} % Degree
    {Équipe d'intégration pour le logiciel Xperthis Care} % Institution
    {} % Location
    {} % Date(s)
    {
    Xperthis Care est le nouveau dossier médical au CHU Tivoli. Il est amené à remplacer un logiciel plus ancien développé en interne et qui était jusqu'ici utilisé quotidiennement par plus de 1000 personnes.
    
    Le but du projet est d'intégrer les données de l'ancien dossier médical vers le nouveau. Ma mission est d'aider l'équipe à analyser, transformer, tester et intégrer les données d'un logiciel à l'autre.

    Technologies utilisées : HL7 / CSV / XML / SAP Sybase / InterSystems Caché / Xperthis Care.

    }
    
%---------------------------------------------------------
  \cventry
    {2016 - Présent} % Degree
    {HL7 - Échange d'informations sur la santé} % Institution
    {} % Location
    {} % Date(s)
    {
    Un hôpital est connecté à un nombre considérable d'appareils médicaux et administratifs. 
    
    Mon rôle est de gérer les échanges d'informations de santé, leurs agrégations ainsi que les flux entre tous les systèmes médicaux de l'hôpital de manière à les rendre disponibles d'un système à un autre.
    Un exemple concret serait un appareil médical qui envoie des résultats vers le dossier médical du patient. 
    
    Il y a actuellement plus de 150 flux d'échange interconnectés via la plateforme InterSystems HealthShare. 

    Technologies utilisées : HL7 / InterSystems Caché / InterSystems HealthShare / Java / SAP Sybase.

    }
    

%---------------------------------------------------------
  \cventry
    {Sep 2019 - Présent} % Degree
    {Xperthis Care Support Team} % Institution
    {} % Location
    {} % Date(s)
    {
    Xperthis Care est le nouveau logiciel au sein de l'hôpital CHU Tivoli. Mon rôle dans cette position est d'aider les utilisateurs à utiliser l'application et ses multiples modules lors de la phase de transition avec l'ancien logiciel médical.

    }
    
%---------------------------------------------------------
  \cventry
    {2008 - Présent} % Degree
    {Application de gestion centraliése des utilisateurs} % Institution
    {} % Location
    {} % Date(s)
    {
    Pour faciliter le traitement des utilisateurs au sein de l'institution, j'ai dû analyser, architecturer, développer et mettre en production un système centralisé pour gérer les données administratives et les droits de plusieurs sous-systèmes et applications (Active Directory, Suprema BioStar, OpenLDAP, Metalprogetti, Microsoft Exchange...). Ceci pour l'ensemble de la société qui compte plus de 2000 utilisateurs.
    
    Cette application est destinée à l'équipe de support qui est responsable de la création et la gestion des utilisateurs dan l'institution. Ils ont par moments plus de 50 utilisateurs à créer à la suite, c'est pour cela que la centralisation de la gestion leur rend la vie plus facile. 
    
    Afin de gagner en rapidité et en fiabilité des données, la lecture de la puce de la carte d'identité belge est faite lors de la création d'un compte utilisateur. 

    Technologies utilisées : Java / CSS / HTML / JavaScript / Apache Tomcat / Python / Linux / Active Directory / openLDAP.

    }
    
    
%---------------------------------------------------------
  \cventry
    {2018 - 2019} % Degree
    {Portail d'authentification à deux facteurs} % Institution
    {} % Location
    {} % Date(s)
    {
    En vue de donner accès à plus de 2500 utilisateurs aux applications internes de l'hôpital, l'institution a décidé d'implémenter les systèmes BIG-IP F5 et Awingu. Cependant, les deux systèmes ne peuvent pas fonctionner en symbiose. 
    
    Mon rôle dans ce projet est de créer une page sécurisée donnant par la suite l'accès à F5 et Awingu. 
    Un système d'authentification à deux facteurs a été mise en place en combinant un développement Java SpringBoot et la configuration d'un serveur NGINX. Lors de la connexion à cette page sécurisée, l'utilisateur est invité à entrer son login, mot de passe et un code qui lui est envoyé par SMS.

    Technologies utilisées : Java / Spring Boot / CSS / HTML / NGINX / Active Directory / SMS / AT Commands.

    }
    
    
    
    
%---------------------------------------------------------
  \cventry
    {Jan 2015 - Dec 2015} % Degree
    {Prescription laboratoire électronique - HL7} % Institution
    {} % Location
    {} % Date(s)
    {
    Développement d'une application de prescription laboratoire de manière à faciliter la traçabilité et la communication entre l'équipe médicale et le laboratoire. 

    Mon rôle est d'architecturer, développer et mettre en production le logiciel pour lequel une interface utilisateur conviviale a été pensée. 

    Les médecins et les infirmières prescrivent des demandes de laboratoire. Ces dernières sont envoyées instantanément de manière électronique au standard HL7 vers Molis, le logiciel du laboratoire.
    Ce flux de données HL7 transite par la plateforme InterSystems HealthShare.
    
    Chaque prescription est stockée dans une base de données SAP Sybase afin de garder un historique pour le staff médical.
    
    Pour réduire le temps d'encodage dans l'application, un système de "profiles" est créé. Un utilisateur peut pré-cocher une série d'analyses utilisées fréquemment et les sauver en vue de les ré-utiliser lors de la prescription à un patient. 

    Technologies utilisées : Java / CSS / HTML / JavaScript / Apache Tomcat / SAP Sybase.

    }
    
    
    
    
%---------------------------------------------------------
  \cventry
    {-} % Degree
    {Déclaration d'événements indésirables} % Institution
    {} % Location
    {} % Date(s)
    {
    Les employés sont tenus de signaler tout événement indésirable qui s'est produit au sein de l'hôpital.
    
    Mon rôle dans ce projet est de développer un logiciel afin de passer d'une version papier à une version électronique de cette déclaration.
    
    J'ai analysé, développé, et mis en production l'application.
    Ceci facilite la gestion quotidienne et la traçabilité des déclarations par l'équipe du service Qualité.
    
    La structure de l'application est organisée autour de la classification de l'OMS et des recommandations de la Belgique au niveau de la santé. Les données gérées peuvent aisément être exportées à destination des autorités fédérales belges.

    Technologies utilisées : Python / Django / HTML5 / JavaScript / MySQL / Linux / Apache HTTP Server.

    }
    
    
%---------------------------------------------------------
  \cventry
    {-} % Degree
    {Développement d'une feuille de style CSS globale à l'hôpital} % Institution
    {} % Location
    {} % Date(s)
    {
    Le dossier médical est développé en interne et ses multiples modules sont programmés par des personnes différentes de l'équipe.
    
    Mon but est d'analyser et de développer une feuille de style commune pour chaque application produite en interne afin de standardiser leur look. L'architecture est basée sur les méthodologies BEM et ITCSS.


    Technologies utilisées : CSS / SASS.

    }
    
    
%---------------------------------------------------------
  \cventry
    {-} % Degree
    {Envoi électronique des résultats patient à leur médecin généraliste} % Institution
    {} % Location
    {} % Date(s)
    {
    Le but du projet est de passer du format papier à un format électronique pour l'envoi des résultats patients à plus de 16 000 médecins généralistes au travers de la Belgique. Ils ont ainsi les résultats de leurs patients plus rapidement.
    
    Mon rôle est d'architecturer, développer, intégrer et mettre en production la solution. Elle est composée d'un mix entre des développements internes, de multiples transformations de données ainsi que l'intégration au logiciel "MediMail". Ce dernier est une boîte mail sécurisée développée par une société spécialisée à laquelle les médecins de Belgique sont connectés. 

    Technologies utilisées : Java / MediMail / Windows Server / XML.

    }
    
   
%---------------------------------------------------------
  \cventry
    {-} % Degree
    {Antécédents médicaux} % Institution
    {} % Location
    {} % Date(s)
    {
    Permettre aux médecins et aux infirmiers d'encoder et gérer les antécédents médicaux des patients dans le dossier médical de l'hôpital. L'ensemble des données importantes sont rassemblées sur une seule interface. En un clin d'oeil, les spécialistes peuvent prendre de meilleures décisions concernant les diagnostiques et les soins à procurer.
    
    Le projet est intégré à un autre module médical développé par mes soins. Voir le point "Vue centralisée du dossier patient".
    
    Analyser, architecturer, développer, concevoir l'interface utilisateur et mettre en production font partie de mes attributions pour le projet. 


    Technologies utilisées : Java / HTML / JavaScript / CSS / Apache Tomcat / Sybase.

    }
    
    
   
%---------------------------------------------------------
  \cventry
    {-} % Degree
    {Vue centralisée du dossier patient} % Institution
    {} % Location
    {} % Date(s)
    {
    Mediweb est le dossier médical patient développé en interne de l'hôpital. Le but du projet est de centraliser les données santé du patient provenant de plusieurs logiciels sous forme d'une vue conviviale et pratique pour l'utilisateur. Le staff médical peut accéder plus facilement et rapidement aux données pertinentes sans devoir naviguer entre plusieurs logiciels médicaux.
    
    Les données peuvent être filtrées par type, auteur, prescripteur, années et visionnées en un click au lieu de chercher le document dans les applications originales qui contiennent l'information.


    Mon rôle est d'architecturer, concevoir l'interface utilisateur, développer et mettre en production la solution.


    Technologies utilisées : Java / HTML / JavaScript / jQuery / CSS / Apache Tomcat / SAP Sybase.

    }
    
    
   
%---------------------------------------------------------
  \cventry
    {-} % Degree
    {Interconnexion d'hôpitaux - Réseau Santé Wallon (RSW)} % Institution
    {} % Location
    {} % Date(s)
    {
    Interconnecter l'hôpital du CHU Tivoli en le reliant au Réseau Santé Wallon qui est le pivot de données de santé entre les hôpitaux de Belgique.
    
    Mon rôle est d'architecturer, développer des Web services SOAP et mettre en production la solution. Grâce à celle-ci, les praticiens peuvent consulter les documents de chaque hôpitaux de Belgique.

    Technologies utilisées : Java / XML / SOAP / Web ervices / Linux / Windows Server.

    }
    
    
   
%---------------------------------------------------------
  \cventry
    {-} % Degree
    {Rappels de rendez-vous par SMS} % Institution
    {} % Location
    {} % Date(s)
    {
    Chaque jour, l'hôpital perd de l'argent parce que des patients ne se présentent pas à leur rendez-vous chez un spécialiste.
    
    Mon objectif est d'architecturer, développer, interconnecter et mettre en production un service d'envoi de rappels par SMS des rendez-vous aux patients quelques jours auparavant. Les patients peuvent annuler leur rendez-vous en envoyant un SMS avec un code qui leur est attribué. L'hôpital peut dès lors attribuer le créneau devenu libre à une autre personne.
    
    Pour l'envoi des rappels, les futurs rendez-vous sont extraits du logiciel d'agenda de l'hôpital. Ils sont par la suite convertis et transférés à la passerelle Foxbox SMS pour l'envoi. Cette étape se fait au travers de la plateforme InterSystems HealthShare.
    Les SMS de retour sont également traités par cette plateforme afin d'expédier l'information vers le logiciel d'agenda.

    Technologies utilisées : Java / InterSystems HealthShare / InterSystems Caché / Foxbox SMS Gateway.

    }
    
    
 
%---------------------------------------------------------
  \cventry
    {-} % Degree
    {Implémentation de la plateforme Suprema BioStar} % Institution
    {} % Location
    {} % Date(s)
    {
    Implémenter la plateforme de lecture biométrique BioStar et ses dispositifs, BioStation et BioEntry Plus au sein de l'hôpital.
    
    Développement de code pour interconnecter le logiciel BioStar et le logiciel SP-Expert afin d'utiliser les appareils BioStation en tant que pointeuses.

Technologies utilisées : Suprema BioStar 1.x / Java / Windows Server / SQL Server.
    }

%---------------------------------------------------------
\end{cventries}
