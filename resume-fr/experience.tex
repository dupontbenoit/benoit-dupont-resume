%-------------------------------------------------------------------------------
%	SECTION TITLE
%-------------------------------------------------------------------------------
\cvsection{Expérience}


%-------------------------------------------------------------------------------
%	CONTENT
%-------------------------------------------------------------------------------
\begin{cventries}

%---------------------------------------------------------
  \cventry
    {Analyste/intégrateur HL7 et développeur Full-Stack} % Job title
    {CHU Tivoli} % Organization
    {La Louvière, Belgique} % Location
    {Août 2008 - Présent} % Date(s)
    {
    Lors de mes 13 ans de carrière au CHU Tivoli, j'ai pu étendre mes connaissances et découvrir les multiples facettes du monde de l'informatique au sein d'une super équipe de 25 personnes.
    Que ce soit du support utilisateur pour 2000+ personnes en passant par l'administration des serveurs jusqu'à la gestion des flux de données médicales HL7 via la plateforme InterSystems HealthShare.
    
    J'ai également eu l'opportunité d'analyser, architecturer, développer et mettre en production un grand nombre d'applications médicales, administratives et autres outils utiles à l'entreprise.
    
    Étant dans le milieu médical, je me suis retrouvé dans le rôle de spécialiste HL7. Dans celui-ci, j'ai eu l'occasion d'analyser les données en vue de mettre en place des interfaces d'intégration entre plusieurs fournisseurs et logiciels médicaux au sein de l'hôpital.
    L'analyse est très importante dans ce type de projet.
    Un standard tel que le HL7 est principalement une enveloppe dans laquelle les fournisseurs et systèmes sont susceptibles d'utiliser des données d'identification différentes. Par exemple, un identifiant patient dans un système n'est pas le même dans un autre. Il faut s'assurer que les deux utilisent la même référence ou mettre en place des transformations de données entre les deux. Ce genre d'analyse et de transformations ont fait partie de mon quotidien pendant plusieurs années que ce soit lors de la mise en place de chaque nouvel appareil médical ou lors de la migration d'un logiciel médical à un autre.
    
    }

%---------------------------------------------------------
  \cventry
    {Spécialiste informatique} % Job title
    {Line Up Team} % Organization
    {La Louvière, Belgique} % Location
    {2019 - Présent} % Date(s)
    {
    Line Up Team est une start-up dans le domaine de la communication pour laquelle j'interviens dans de nombreux domaines de l'informatique.
    
    J'ai développé et mis en production des outils permettant de gérer les statistiques des réseaux sociaux de leurs clients. L'application est développée en Python/Django et est déployée sur un serveur VPS que je maintiens.
    
    Line Up Team héberge les sites web WordPress et les courriels de leurs clients. Mon rôle dans ce cas est de configurer les serveurs afin de fournir les services à leurs nouveaux clients. La manipulation de cPanel et de la ligne de commande "wp cli" est un plus pour standardiser les manipulations.

    }
    
    
%---------------------------------------------------------
  \cventry
    {Développeur Full-Stack} % Job title
    {SPA La Louvière (Bénévole)} % Organization
    {La Louvière, Belgique} % Location
    {2019 - Présent} % Date(s)
    {
    La gestion du refuge animalier était basée sur de multiples feuilles Excel. J'ai proposé de développer un logiciel simple à utiliser afin de leur faciliter la gestion quotidienne.

    C'est en me basant sur mon expérience de développeur de logiciels médicaux au CHU Tivoli que j'ai pu leur fournir une application qui leur facilite la vie et leur fait gagner du temps.
    
    Cette app est développée en Python/Django et tourne sur un serveur VPS que je maintiens. 
    
    J'aide également l'équipe avec d'autres tâches informatiques telles que le développement de plugin et les mises à jour de leur site WordPress.

    }

%---------------------------------------------------------
  \cventry
    {Développeur Web} % Job title
    {URLC} % Organization
    {La Louvière, Belgique} % Location
    {Jan 2012 - Sep 2019} % Date(s)
    {
    Développement d'un CMS (basé sur la version précédente pour www.raal.be) pour gérer dynamiquement le site web www.urlc.be.
    Évolutions constantes en fonction des besoins.

    Technologies utilisées : PHP / HTML / CSS / JavaScript / jQuery / MySQL.

    }

%---------------------------------------------------------
  \cventry
    {Web Developer} % Job title
    {RAA Louviéroise} % Organization
    {La Louvière, Belgique} % Location
    {2001 - 2006} % Date(s)
    {
    En 2001 WordPress n'existait pas encore. La mise à jour de sites web était fastidieuse. J'ai développé un CMS afin de gérer facilement et dynamiquement le site web www.raal.be

Technologies utilisées : PHP / HTML / CSS / Javascript / jQuery / MySQL.

    }
\end{cventries}
