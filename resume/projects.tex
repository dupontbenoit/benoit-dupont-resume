%-------------------------------------------------------------------------------
%	SECTION TITLE
%-------------------------------------------------------------------------------
\cvsection{Projects}


%-------------------------------------------------------------------------------
%	CONTENT
%-------------------------------------------------------------------------------
\begin{cventries}

%---------------------------------------------------------
  \cventry
    {Sep 2019 - Present} % Degree
    {Xperthis Care Support Team} % Institution
    {} % Location
    {} % Date(s)
    {
Xperthis Care is the new medical software at CHU Tivoli destined to replace an older one.

My role in this position was to help the end-users using the software and its multiple modules.

    }
    
%---------------------------------------------------------
  \cventry
    {2019 - Present} % Degree
    {Xperthis Care Development Team} % Institution
    {} % Location
    {} % Date(s)
    {
Xperthis Care is a great medical record software but is lacking some functionalities needed by the medical staff. We had to develop add-ons using InterSystems Caché programming language to better satisfy the users' needs.

One of them was to integrate the physicians' agenda into Care so they don't have to switch between multiple applications. I had to analyze, architect, and develop the project.

Tech stack: InterSystems Caché / HTML / CSS / Angular / TypeScript / GitLab CI

    }
    
    
%---------------------------------------------------------
  \cventry
    {2019 - Present} % Degree
    {Xperthis Care Integration Team} % Institution
    {} % Location
    {} % Date(s)
    {
Xpethis Care is the new medical record at CHU Tivoli. It's replacing an older software developed in-house used by 1000+ people.

We had to make integrate data from the old medical record to the new one. My mission was to help the team analyzing, transforming, testing, and integrating the data from one medical record to the other one.

Tech stack: flat files / CSV files / XML files / SAP Sybase / InterSystems Caché / Xperthis Care / HL7

    }
    
%---------------------------------------------------------
  \cventry
    {2016 - Present} % Degree
    {HL7 Health Information Exchange} % Institution
    {} % Location
    {} % Date(s)
    {
A hospital is connected to a lot of administrative and medical devices and software. 
My role is to manage health information exchange, data aggregation, and workflow between all medical systems in the hospital so they are available from one system to another. One example would be a medical device sending data to the patient's medical record. 

150+ workflow are connected to each other in InterSystems HealthShare's production server.

Tech stack: HL7 / InterSystems Caché / InterSystems HealthShare / Java / SAP Sybase

    }
    
%---------------------------------------------------------
  \cventry
    {2008 - Present} % Degree
    {Centralized user management} % Institution
    {} % Location
    {} % Date(s)
    {
For easier user management I had to analyze, architect, develop, and put in production a centralized system to manage the 2000+ users' administrative data and credentials for multiple subsystems and apps (Active Directory, Suprema BioStar, OpenLDAP, Metalprogetti, Microsoft Exchange...)

This app is designed for the support team who is responsible for the user creation and management in the institution. They have multiple new users every day. Twice a year they have 50+ students to create at the same time that's where the system shines by being centralized for managing multiple apps.
For faster form completion, most of the needed data is read from the Belgium Identity Card's microchip.

Tech stack: Java / CSS / HTML / JavaScript / Apache Tomcat / Python / Linux / Active Directory

    }
    
    
%---------------------------------------------------------
  \cventry
    {2018 - 2019} % Degree
    {2FA portal} % Institution
    {} % Location
    {} % Date(s)
    {
To allow the 2500+ users to connect to hospital internal web apps the institution implemented BIG-IP F5 and Awingu for remote desktop. 
The two systems cannot work with each other.
My role in this project is to build a secured front page to access these systems.

With a combination of NGINX configuration and Java SpringBoot code I managed to develop login system with 2 factors authentication. The user goes to a URL, login and password are required, an SMS is sent to the user to allow access to the BIG-IP F5 and Awingu systems. 

The Java code is connected to the institution's Active Directory to check the credentials. Then it uses a TCP/IP connection to a physical SMS Router to send the SMS using AT Commands.

Tech stack: Java / Spring Boot / CSS / HTML / NGINX / Active Directory / SMS / AT Commands

    }
    
    
    
    
%---------------------------------------------------------
  \cventry
    {Jan 2015 - Dec 2015} % Degree
    {Laboratory Prescription} % Institution
    {} % Location
    {} % Date(s)
    {
Development of an app for easier traceability and faster communication between the lab and the medical team.

My role is to architect, develop, and put in production the app packed into a user-friendly web interface. 
Physicians and nurses can prescribe labs requests. The request is dispatched instantly to the lab in HL7 standard to Molis lab software. Data goes through InterSystems HealthShare for connectivity. 
Every prescription is stored in the SAP Sybase database and can be reviewed at any time by the users.

For faster request creation, I developed a "save request profile". The medical staff can save and reuse the same pre-checked analysis multiple times.

Tech stack: Java / CSS / HTML / JavaScript / Apache Tomcat / SAP Sybase

    }
    
    
    
    
%---------------------------------------------------------
  \cventry
    {-} % Degree
    {Adverse events report} % Institution
    {} % Location
    {} % Date(s)
    {
Practitioners and nurses have to report adverse events concerning a patient to the authorities.
My role is to go from paper-based form to electronic data.
I architected, developed, and put in production an app for easier management by the Quality Team.
Competent authorities can access all the reports, manage the cases, and extract statistics.
The organizational structure of the app is based on OMS and Belgium Health recommendations.
The data can be extracted to be sent to Belgium federal authorities when needed.

Tech stack: Python / Django / HTML5 / JavaScript / MySQL / Linux / Apache HTTP Server

    }
    
    
%---------------------------------------------------------
  \cventry
    {-} % Degree
    {CSS Stylesheet Framework development} % Institution
    {} % Location
    {} % Date(s)
    {
The medical record is developed in-house and is composed of multiple modules each developed by a member of the team.
My goal is to analyze and develop a CSS Stylesheet used as a Framework for future modules and other internal software developments.
The architecture is based on BEM and ITCSS methodologies. 

Tech stack: CSS / SASS

    }
    
    
%---------------------------------------------------------
  \cventry
    {-} % Degree
    {Electronic sending of patient's medical records} % Institution
    {} % Location
    {} % Date(s)
    {
The goal of the project is to move from paper to electronic data to send medical results to the 16000+ practitioners in Belgium so they can access their patients' results faster.

My role was to architect, develop, integrate and put in production the solution. It's a mix between internal app development, data transformation, and integration with software named MediMail, a secured medical mailbox for practitioners. 

Tech stack: Java / MediMail / Windows Server / XML

    }
    
   
%---------------------------------------------------------
  \cventry
    {-} % Degree
    {Medical History} % Institution
    {} % Location
    {} % Date(s)
    {
Let's physicians and nurses add medical history to the patient's medical record so specialists can make better decisions in choosing the right care or diagnosis.

Then project is integrated with another module I have developed (see below Mediweb patient's central medical record page)

In a glimpse of an eye, the medical staff has access to all the relevant and important results regarding the patient.

I had to architect, develop, UI design and put in production the solution developed.

Tech stack: Java / HTML / JavaScript / CSS / Apache Tomcat / Sybase

    }
    
    
   
%---------------------------------------------------------
  \cventry
    {-} % Degree
    {Mediweb patient's central medical record page} % Institution
    {} % Location
    {} % Date(s)
    {
Mediweb is the internal medical software. The project's goal was to centralize a patient's medical record from all the sources we have in the hospital in one user-friendly application. It allows the medical staff to access relevant data more easily and more quickly than going through multiple applications. 

The user can filter data by source, author, prescriber, date range and open the document in one click instead of looking in each app to find the document they need to complete the diagnosis.

My role was to architect, UI design, develop and put in production the solution.

Tech stack: Java / HTML / JavaScript / jQuery / CSS / Apache Tomcat / Sybase

    }
    
    
   
%---------------------------------------------------------
  \cventry
    {-} % Degree
    {Réseau Santé Wallon (RSW)} % Institution
    {} % Location
    {} % Date(s)
    {
Interconnection between CHU Tivoli and Réseau Santé Walloon central medical record hub between Walloon Region hospitals.

My role was to architect, develop and put in production the solution so the hospital can be connected to the RSW. From there, each practitioner in Belgium can access results from hospitals.

Tech stack: Java / XML / SOAP / Linux

    }
    
    
   
%---------------------------------------------------------
  \cventry
    {-} % Degree
    {SMS rendezvous reminder} % Institution
    {} % Location
    {} % Date(s)
    {
Each day the hospital is losing money because patients don't show up to their appointment.

My goal is to architect, develop, interconnect, and put in production an SMS service to send reminders to patients few days before their appointment. Patients can respond to the SMS with a special provided code to cancel the appointment if needed so the hospital can re-schedule the slot for another patient.

Futur appointments are extracted in XML from the hospital agenda, sent through InterSystems HealthShare, and then converted and sent to the Foxbox SMS Gateway for sending. Another system is responsible to take care of patients responses and cancel the appointment if needed. 

Tech stack: Java / InterSystems HealthShare / InterSystems Caché / Foxbox SMS Gateway

    }
    
    
 
%---------------------------------------------------------
  \cventry
    {-} % Degree
    {Suprema BioStar implementation} % Institution
    {} % Location
    {} % Date(s)
    {
Implementing BioStar biometric technology for access control with BioStation and BioEntry Plus devices.

Interconnection between BioStar software and SP-Expert software to use BioStations as time and attendance device. 

Tech stack : Suprema BioStar 1.x / Java / Windows Server / SQL Server
    }

%---------------------------------------------------------
\end{cventries}
